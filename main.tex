\documentclass[conference]{IEEEtran}
\IEEEoverridecommandlockouts
% The preceding line is only needed to identify funding in the first footnote. If that is unneeded, please comment it out.
\usepackage{cite}
\usepackage{amsmath,amssymb,amsfonts}
\usepackage{algorithmic}
\usepackage{graphicx}
\usepackage{textcomp}
\usepackage{xcolor}
\usepackage{ctex}
\def\BibTeX{{\rm B\kern-.05em{\sc i\kern-.025em b}\kern-.08em
    T\kern-.1667em\lower.7ex\hbox{E}\kern-.125emX}}



\begin{document}
% \bibliographystyle{plain}

\title{图神经网络在动态图上的链路预测综述}

\author{
	\IEEEauthorblockN{王天赐}
	\IEEEauthorblockA{
		\textit{计算机学院} \\
		\textit{湖北工业大学} \\
		武汉,湖北,中国 \\
	}
}

\maketitle



\begin{abstract}
图形神经网络(GNNs)正迅速成为在图形结构数据上学习的主导方式。链接预测是新的GNN模型的一个几乎通用的基准。许多先进的模型,如动态图神经网络(DGNNs)专门针对动态链接预测。然而,这些模型,特别是DGNNs,很少与其他模型或现有启发式方法进行比较。不同的工作以不同的方式评估他们的模型,因此人们无法直接比较评估指标。受此启发,我们进行了一次全面的比较研究。我们比较了链接预测启发式方法、GNNs、离散DGNNs和连续DGNNs对动态链接预测的作用。我们发现,简单的链接预测启发式方法往往比GNN和DGNN表现更好,而在所有被考察的图神经网络中,DGNN的表现一直优于静态GNN。
\end{abstract}

\begin{IEEEkeywords}
动态图神经网络; 启发式; 链接预测;
\end{IEEEkeywords}

\section{引言}
\subsection{研究背景}
在本文中主要的研究内容是在动态图上的链接预测, 与静态图的链接预测不同的是, 动态图的链接预测增加了时间维度, 预测难度增加, 但是在现实中的应用更加广泛。

\subsection{研究方法}
在近年来的研究中, 很多链路预测方法取得了发展, 在静态图上研究者们提出了很多高精度的链接预测方法。比如基于启发式算法的共同邻居算法\cite{zhang2008recommendation}, 
以及首次将GNN应用在图的链接预测的SEAL模型\cite{zhang2008recommendation}, SEAL模型

\subsection{研究问题}

近年来,图神经网络(GNNs)作为一个新兴的研 究领域,得到了长足的发展,提出并发展了多种体系结构。但是在动态图神经网络(DGNN)领域,这些问题因以下原因而进一步加剧:

\subsubsection{数据的动态性质}
\subsubsection{缺乏通用术语}
\subsubsection{缺乏既定的强大基线(大多数研究不与其他DGNN比较性能)}
\subsubsection{离散和连续DGNN之间的鸿沟}
\subsubsection{大量的实验设计选择}
这些选择包括:如何表示动态网络(如快照、时间窗口、连续、边的生存时间等),包括哪些节点特征,如何将数据分成训练-验证-测试集,用哪些指标来评价结果,如何在报告的指标中使用负采样率,以及如何选择/优化神经网络参数(如学习率、早期停止准则、嵌入空间维度等)。所有这些都意味着,通过阅读研究论文来比较方法的性能是不可能的,除非他们清楚地说明他们所有的设计选择,而且这些设计选择在不同的论文中是相同的。
这些选择包括:如何表示动态网络(如快照、时间窗口、连续、边的生存时间等),包括哪些节点特征,如何将数据分成训练-验证-测试集,用哪些指标来评价结果,如何在报告的指标中使用负采样率,以及如何选择/优化神经网络参数(如学习率、早期停止准则、嵌入空间维度等)。所有这些都意味着,通过阅读研究论文来比较方法的性能是不可能的,除非他们清楚地说明他们所有的设计选择,而且这些设计选择在不同的论文中是相同的。
DGNNs是建立网络动态模型的一个很有前途的途径,因为它们既能通过GNNs编码空间模式,又能通过时间序列组件(如循环神经网络(RNN)或自我注意)编码时间模式。然而,迄今为止提出的DGNNs已经在少数数据集上进行了测试,并且很少与其他DGNNs进行比较。不同的研究在不同的数据集上进行比较,因为在DGNN基准测试中使用哪种数据集的问题上没有共识。

\section{相关研究模型与数据集}

\subsection{相关数据集}
我们选择了五个连续的交互网络和一个离散的演化网络(Autonomous)作为数据集。我们选择了互动网络,因为它们允许我们轻松地转换为更粗粒度的时间粒度,如离散网络。更稀疏的快照表明链接和非链接之间有更大的不平衡性,从而使分类问题更难。我们为每个数据集准备了两个版本,一个是有方向的连续交互网络,一个是无方向的离散网络。连续模型对连续网络进行编码。静态和离散模型对离散网络进行编码。在从连续到离散的转换中,互换的边缘被添加到离散网络中,使其成为无定向的。一条边在快照中出现的次数被作为权重加到快照的边上。
所有的结果都是对离散网络的预测报告。对于连续模型,这是通过将连续网络的连续部分分割成与离散网络中的快照相对应的快照来实现的。然后,我们让连续模型在目标快照前对连续网络进行编码,然后尝试预测离散网络中的链接发生。
在本文中,我们使用了以下数据集:

\subsubsection{Enron}
这个数据集是一个电子邮件通信网络,其中一个链接是两个人之间发送的电子邮件。Enron在空间上是一个很小的网络,但在时间上是一个中等规模的网络,有合理数量的连续链接,覆盖的时间跨度超过3年。
由于节点数量少,边的数量相对较多,它比其他网络要密集得多。

\subsection{UC Irvine messages}
简称UC,是加州大学欧文分校的一个在线论坛网络。如果两个学生在同一个论坛帖子上互动,他们就会被连接起来。
因此,这原本是一个二方网络,但它被预测为只有一种类型的节点。快照大小的奇特选择来自于EvolveGCN(Pareja等人,2020),它观察到较小的快照大小会产生一些没有任何边的快照

\subsection{相关研究模型}


 
\begin{thebibliography}{00}
\bibitem{zhang2008recommendation}
J.~Zhang, J.~Tang, B.~Liang, Z.~Yang, S.~Wang, J.~Zuo, and J.~Li,
``Recommendation over a heterogeneous social network,'' in \emph{Web-Age
	Information Management, 2008. WAIM'08. The Ninth International Conference
	on}.\hskip 1em plus 0.5em minus 0.4em\relax IEEE, 2008, pp. 309--316.

\bibitem{Ronneberger2015U}
O.~Ronneberger, P.~Fischer, and T.~Brox, \emph{U-Net: Convolutional Networks
	for Biomedical Image Segmentation}.\hskip 1em plus 0.5em minus 0.4em\relax
Springer International Publishing, 2015.
\end{thebibliography}

% \bibliography{ref}
\end{document}
